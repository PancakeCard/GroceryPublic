%笔记
%一个latex只能有一个document环境,即\begin{环境}
%空行为换行
%处理中文要求文档为UTF-8格式
%ctex宏包说明>texdoc ctex
%简单教程>texdoc lshort-zh
%texstudio ctrl+T注释,ctrl+U取消注释


%导言区(全局设置)
\documentclass{article}
%\documentclass{ctexart}
%\documentclass[10pt]{article}
%article,book,report,letter|ctexart
%10pt是参数,表示normal字体大小
\usepackage{ctex}%中文支持
\newcommand\degree{^\circ}%自定义命令,用于支持下面的\degree命令
\title{文件基本结构}
\author{ZJB}
\date{\today}

%正文区
\begin{document}
	%使标题信息生效
	\maketitle
	%普通文本
	Hello world!
	
	你好,\LaTeX!
	
	%行内公式
	inline formula, $f(x)=3x^2+x-1$,$g(x)=\frac{1}{2}x$ and so on.
	
	%块公式
	block formula,$$\tanh(z)=\frac{e^z-e^{-z}}{e^z+e^{-z}}$$.
	
	%带编号公式
	%degree自定义命令
	\begin{equation}
	AB^2=BC^2+AC^2,\angle C=90\degree
	\end{equation}
	
	%字体部分,大括号可以约束范围
	%字体族设置(罗马字体、无衬线字体、打字机字体)
	\textrm{Roman Family}
	\textsf{Sans Serif Family}
	\texttt{Typewriter Family}
	
	{\rmfamily Roman Family}
	{\sffamily Sans Serif Family}
	\ttfamily Typewriter Family
	
	The next paragraph is Typewriter Family too.
	\rmfamily%不限制范围影响后续文本
	
	%字体形状(直立、斜体、伪斜体、小型大写)
	\textup{Upright Shape} \textit{Italic Shape} \textsl{Slanted Shape} \textsc{Small Caps Shape}
	
	{\upshape Upright Shape} {\itshape Italic Shape} {\slshape Slanted Shape} {\scshape Small Caps Shape}
	
	%字体设置(粗细、宽度)
	\textmd{Medium Series}
	\textbf{Boldface Series}
	
	{\mdseries Medium Series}
	{\bfseries Boldface Series}
	
	%中文字体,需要ctex宏包
	{\songti 宋体} \quad {\heiti 黑体} \quad {\fangsong 仿宋} \quad {\kaishu 楷书}
	
	中文\textbf{粗体}和\textit{斜体}
	
	%字体大小
 	{\tiny 	        Hello}
	{\scriptsize    Hello}
	{\footnotesize  Hello}
	{\small         Hello}
	{\normalsize    Hello}
	{\large         Hello}
	{\LARGE         Hello}
	{\huge          Hello}
	{\Huge          Hello}
	
	%中文字大小
	调整前
	\zihao{4}
	调整后
	\zihao{5}
	调整回
	
	\newpage
	%特殊字符
	\section{空白符号}
	Multiple spaces is same as              one;
	
	中文内的空                          格不显示;
	
	中英混排时chinese and english之间自动插入空格;
	
	
	
	多个空行等同于一个;
	
	自动缩进,不能用空格代替;
	
	手动空格用下列语句:
	
	%参考字符
	一二三四五六七
	
	%1em
	啊\quad 啊quad
	
	%2em
	啊\qquad 啊qquad
	
	%1/6em
	啊\thinspace 啊\,啊thinspace
	
	%0.5em
	啊\enspace 啊enspace
	
	%空格
	啊\ \ 啊\ 空格*2
	
	%手动设置
	啊\kern 1em 啊kern
	
	啊\hskip 1em 啊hskip
	
	啊\hskip 2em 啊hskip
	
	%占位宽度
	啊\hphantom{xyz} 啊hphantom
	
	%弹性长度
	啊\hfill 啊hfill
	\section{控制符}
	\# \$ \% \{ \} \~{} \_{} \^{} \textbackslash \&
	\section{排版符号}
	\S \P \dag \ddag \copyright \pounds
	\section{\TeX 标志符号}
	\TeX{} \LaTeX{} \LaTeXe{}
	\section{引号}
	` ' `` ''
	\section{连字符}
	- -- ---
	\section{非英文字符}
	\oe \OE \ae \AE \aa \AA \o \O \l \L \ss \SS !` ?`
	\section{重音符号(o为例)}
	\`o \'o \^o \''o \~o \=o \.o \u{o} \v{o} \H{o} \r{o} \t{o} \b{o} \c{o} \d{o}
\end{document}