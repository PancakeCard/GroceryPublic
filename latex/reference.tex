%参考文献管理,bibtex
\documentclass{ctexart}

%注意每次换样式要清理辅助文件,工具——清理辅助文件。
%除本文内容外,教程还提到了新的TEX参考文献排版引擎——biber,其支持本地化排版。

%排版格式
%\usepackage[round]{natbib}

%参考文献样式
\bibliographystyle{plain}%plain unsrt alpha abbrv

\begin{document}
	\section{一次管理——正文}
	形状匹配通常有特征提取、相似性度量和度量学习三个步骤的内容\cite{key1}
	
	大数据为地图学带来新的研究内容\cite{key2}
	%一次管理,一次实现
	\begin{thebibliography}{99}
		\bibitem{key1}周瑜,刘俊涛,白翔.\emph{形状匹配方法研究与展望}[J].自动化学报,2012,38(06):889-910.
		
		\bibitem{key2}艾廷华.大数据驱动下的地图学发展[J].测绘地理信息,2016,41(02):1-7.
	\end{thebibliography}

	%可用Google Scholar、Zotero浏览器插件(用于知网文献导出)等便于得到bib文件	
	\section{bib文件文献——正文}
	这里为使用bib文件进行引用,具体格式可能需要再调整
	this is a paper cite by bib file\cite{Ai2007}.
	
	%参考文献路径
	\bibliography{reference/test.bib}
\end{document}